%-----------------------------------------------------------------------------%
\section*{\sc Talks and Presentations}
%-----------------------------------------------------------------------------%
\subsection*{Invited talks}
%-----------------------------------------------------------------------------%
\begin{dated}
	\item[2017]
		\textbf{The functional heterogeneity of interrogative sentences: An `optimistic' approach.}
		2nd workshop on Inquisitiveness Below and Beyond the Sentence Boundary (InqBnB2), 
		Universiteit van Amsterdam,
		December 18--19, 2017.
	\item[2017]
		\textbf{Biscuit conditionals, questions, and biscuit-conditional
questions}, 
		Biscuit Conditionals workshop, 
		University of Hamburg, 
		October 20--21, 2017.
	\item[2017]
		\textbf{Prolegomena to a unified theory of interrogative force}, 
		2nd Speech Act workshop, 
		ZAS, Berlin,
		May 29--31, 2017.
	\item[2016]
		\textbf{(Un)conditional imperatives, (un)conditional modals, and (un)conditional endorsement}, 
		Seminar f\"ur Sprachwissenschaft, 
		Eberhard Karls Universit\"at T\"ubingen, 
		June 6, 2016.
	\item[2015]
		\textbf{Speech act operators vs. extra-compositional conventions of use: What are the issues?}, 
		Speech Act Workshop, 
		ZAS, Berlin, 
		June 11--13, 2015.
	\item[2015]
		\textbf{Conditionalized modal sentences: Modus ponens and Strengthening of the antecedent}, 
		Interdisciplinary Logic Colloquium, 
		University of Konstanz, 
		February 5, 2015.
	\item[2015]
		\textbf{Exclamatives: The conventional dynamic effect of an `expressive' sentence type},
		Colloquium of the Department of Linguistics, 
		University of Konstanz, 
		Januar 22, 2015.
	\item[2015]
		\textbf{Doing things with words: The case of exclamatives}, 
		Department of English, 
		University of G\"ottingen, 
		January 12, 2015.
	\item[2011]
		\textbf{You can't always want what you want: Understanding Anankastic conditionals}, 
		Berkeley Syntax and Semantics Circle, 
		University of California, Berkeley, 
		October 21, 2011.
	\item[2011]
		\textbf{Varieties of compositionality and loose talk}, 
		Department of Linguistics, 
		Johann Wolfgang Goethe-Universit\"at, 
		Frankfurt am Main, Germany, 
		September 28, 2011.
\end{dated}
%-----------------------------------------------------------------------------%
\subsection*{Refereed conferences and workshops}
%-----------------------------------------------------------------------------%
\begin{dated}
	\item[2018]
		with Erlinde Meertens:
		\textbf{Crosslinguistic variation in a minor sentence type: Melioratives in Dutch and German},
		Workshop on non-canonical imperatives,
		Humboldt-Universität zu Berlin, Germany,
		May 25 - 26, 2018.
	\item[2018]
		with Sunwoo Jeong: 
		\textbf{The functional heterogeneity of interrogatives: An `optimistic' approach.}
		{Why Indeed? Questions at the Interface of Theoretical and Computational Linguistics}, 
		workshop at the \textit{40th Annual Conference of the DGfS}, 
		University of Stuttgart, 
		March 7 - 9, 2018.
	\item[2017]
		\textbf{`I believe' in a ranking-theoretic analysis of `believe'}, 
		Twenty-first Amsterdam Colloquium, Amsterdam, The Netherlands,
		December 20--22, 2017.
	\item[2016]
		with Anna Czypionka: 
		\textbf{Temporal implicatures and world knowledge interact rapidly during sentence comprehension}, 
		Logic in Language and in Conversation (LogiCon), 
		Utrecht University, 
		September 19--20, 2016 (poster).
	\item[2016]
		with Prerna Nadathur: 
		\textbf{Quantified indicative conditionals and the relative reading of "most"}, 
		Sinn und Bedeutung 21, 
		University of Edinburgh, 
		September 4--6, 2016.
	\item[2016]
		with Anna Czypionka: 
		\textbf{This week, but not next: Temporal implicatures and their interplay with world knowledge during language comprehension}, 
		Architectures and Mechanisms for Language Processing (AMLaP), 
		Bilbao, 
		September 1--3, 2016 (poster).
	\item[2016]
		\textbf{On the status of `Maximize Presupposition'}, 
		Semantics and Linguistic Theory (SALT) 26, 
		University of Texas at Austin, 
		May 12--15, 2016.
	\item[2015]
		\textbf{Performative uses and the temporal interpretation of modals}, 
		Twentieth Amsterdam Colloquium, 
		Amsterdam, The Netherlands, 
		December 16--18, 2015.
	\item[2015]
		with Cleo Condoravdi: \textbf{Hypothetical facts and hypothetical ideals in the temporal dimension}, 
		{Modelling Modality}, 
		workshop at the 37th Annual Meeting of the Deutsche Gesellschaft für Sprachwissenschaft (DGfS), 
		Leipzig, Germany, 
		March 4-6, 2015.
	\item[2014]
		 \textbf{Biscuits and provisos: Conveying unconditional information by conditional means}, 
		 {Sinn und Bedeutung 19}, 
		 University of G\"ottingen, 
		 September 15--17, 2014.
	\item[2014]
		with Alex Djalali: \textbf{A conceptual-epistemic perspective on model theory}, 
		{Models in Formal Semantics and Pragmatics}, 
		workshop at the 26th European Summer School in Logic, Language and Information (ESSLLI),
		University of T\"ubingen, 
		August 18--22, 2014.
	\item[2014]
		\textbf{Mandatory implicatures in Gricean pragmatics}, 
		{Formal and Experimental Pragmatics}, 
		workshop at the 26th European Summer School in Logic, Language and Information (ESSLLI),
		University of T\"ubingen, 
		August 11--15, 2014.
	\item[2013]
		with Cleo Condoravdi: 
		\textbf{Preference-Conditioned necessities: Anankastic and related conditionals}, 
		{USC Deontic Modality workshop}, 
		University of Southern Calfiornia, 
		May 20--22, 2013.
	\item[2012]
		with Anna Chernilovskaya and Cleo Condoravdi: 
		\textbf{How to express yourself: On the discourse effect of exclamatives}, 
		30th West Coast Conference on Formal Linguistics (WCCFL), 
		University of California, Santa Cruz, 
		April 13--15, 2012.
	\item[2011]
		Alex Djalali, Sven Lauer and Christopher Potts: \textbf{Corpora-driven pragmatics and reference resolution}, Eighteenth Amsterdam Colloquium,
		Amsterdam, The Netherlands, 
		December 19--21, 2011 (poster).
	\item[2011]
		 \textbf{On the pragmatics of pragmatic slack}, 
		 Sinn und Bedeutung 16, 
		 Universiteit Utrecht, The Netherlands, 
		 September 6--8, 2011.
	\item[2011]
		\textbf{Necessity and sufficiency in the semantics of English periphrastic causatives},
		Annual Meeting of the Linguistic Society of America (LSA), 
		Pittsburgh, PA, 
		January 6--9, 2011.
	\item[2010]
		\textbf{Some news on {\it irgendein} and {\it alg\'un}}, 
		Workshop on Epistemic Indefinites,
	    University of G\"ottingen, Germany, 
	    June 10--12, 2010.
	\item[2010]
		with Cleo Condoravdi: 
		\textbf{Performative verbs and performative acts},
		Sinn und Bedeutung (SuB) 15, 
		Saarland University, Germany, 
		September 9--11, 2010.
	\item[2010]
		David Clausen, Alex Djalali, Scott Grimm, Sven Lauer, Tania Rojas-Esponda, and Beth Levin: 
		\textbf{Extension, ontological type, and morphosyntactic class: Three ingredients of countability}, 
		Conference on Empirical, Theoretical and Computational Approaches to Countability in Natural Language, 
		Ruhr-Universit\"at Bochum, Germany, 
		September 22--24, 2010.
	\item[2009]
		\textbf{The `-ever' in `whatever': What, and how?} 
		Workshop on expressives and other kinds of non-truth-conditional meaning,
		held as part of the 31st Annual Meeting of the {Deutsche Gesellschaft f\"ur Sprachwissenschaft} (DGfS), 
		Osnabr\"uck, Germany, March 3--6, 2009.
\end{dated}
%-----------------------------------------------------------------------------%
\subsection*{Selected other talks}
%-----------------------------------------------------------------------------%
\begin{dated}
	\item[2017]
		\textbf{Moore's paradox and hedging with `I Believe': An attempt.} 
		\textit{Questioning Speech Acts}, 
		University of Konstanz, 
		September 14 - 16, 2017.
	\item[2017]
		with Prerna Nadathur: 
		\textbf{Causal necessity, causal sufficiency, and the meaning of causative verbs}, 
		Workshop on modeling causality in formal semantics, 
		University of Konstanz, 
		May 22, 2017.
	\item[2016]
		with Prerna Nadathur: 
		\textbf{Quantified indicative conditionals and the relative reading of {\it most}},
		California Universities Semantics and Pragmatics (CUSP) 9, 
		University of California, Santa Cruz,
		October 21--22, 2016.
	\item[2015]
		\textbf{Temporal interpretation and the performative use of modals}, 
		California Universities Semantics and Pragmatics (CUSP) 8, 
		Stanford University, 
		November 6--7, 2015.
	\item[2014]
		\textbf{Formal semantics, formal pragmatics, and the meaning of desire predicates}, 
		Jour Fixe, Zukunftskolleg, 
		University of Konstanz, 
		May 8, 2014.
	\item[2012]
		\textbf{Implicature tests, and how to fail them: A class of `obligatory' Gricean 
		conversational implicatures}, 
		California Universities Semantics and Pragmatics (CUSP) 5, 
		University of Califoria, San Diego, 
		October 27--28, 2012.
	\item[2012]
		with Cleo Condoravdi: 
		\textbf{Anankastic Conditionals are just conditionals}, 
		California Universities Semantics and Pragmatics (CUSP) 4, 
		University of Southern California, Los Angeles, 
		February 3--4, 2012.
	\item[2011]
		\textbf{Making things happen: Sufficiency causatives in English and Japanese (and German?)},
		Workshop on aspect and modality in lexical semantics, 
		Universit\"at Stuttgart, Germany,  
		September 30, 2011.
	\item[2011]
		with Cleo Condoravdi: 
		\textbf{Towards a null theory of explicit performatives}, 
		Speaking of Possibility and Time II: 1st Workshop on Evidence and Inference, 
		University of G\"ottingen, Germany, 
		June 3--4, 2011.
	\item[2010]
		 \textbf{A pragmatic account of loose talk}, 
		 California Universities Semantics and Pragmatics (CUSP) 3, 
		 Stanford University, 
		 October 15--16, 2010.
	\item[2010]
		with Cleo Condoravdi: 
		\textbf{Imperatives and public commitments}, 
		11th Annual Semantics Fest, 
		Stanford University, 
		March 12, 2010.
	\item[2009]
		with Cleo Condoravdi: 
		\textbf{Performing a wish: Desiderative assertions and performativity},
		California University Semantics and Pragmatics (CUSP) 2, 
		University of California, Santa Cruz, 
		November 21, 2009.
	\item[2009]
		\textbf{`Causative' {\it make}: a veridical, metaphysical conditional?} 
		Workshop on Counterfactuals, Causality and Inertia. 
		Stanford University, 
		August 4, 2010.
	\item[2009]
		\textbf{Free choice of the {\it irgend}-kind: Not as wide as you might think}, California Universities Semantics and Pragmatics (CUSP) 1, 
		University of California, Los Angeles,  
		May 23-24, 2009.
\end{dated}


